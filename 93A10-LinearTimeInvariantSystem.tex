\documentclass[12pt]{article}
\usepackage{pmmeta}
\pmcanonicalname{LinearTimeInvariantSystem}
\pmcreated{2013-03-22 14:22:25}
\pmmodified{2013-03-22 14:22:25}
\pmowner{Mathprof}{13753}
\pmmodifier{Mathprof}{13753}
\pmtitle{linear time invariant system}
\pmrecord{11}{35864}
\pmprivacy{1}
\pmauthor{Mathprof}{13753}
\pmtype{Definition}
\pmcomment{trigger rebuild}
\pmclassification{msc}{93A10}
\pmsynonym{LTI}{LinearTimeInvariantSystem}
%\pmkeywords{LTI}
\pmrelated{Controllability}
\pmrelated{Observability}
\pmrelated{SystemDefinitions}

% this is the default PlanetMath preamble.  as your knowledge
% of TeX increases, you will probably want to edit this, but
% it should be fine as is for beginners.

% almost certainly you want these
\usepackage{amssymb}
\usepackage{amsmath}
\usepackage{amsfonts}

% used for TeXing text within eps files
%\usepackage{psfrag}
% need this for including graphics (\includegraphics)
%\usepackage{graphicx}
% for neatly defining theorems and propositions
%\usepackage{amsthm}
% making logically defined graphics
%%%\usepackage{xypic}

% there are many more packages, add them here as you need them

% define commands here
\begin{document}
A \emph{linear time invariant  system} (LTI) is a linear dynamical system $T(p)$,

\begin{align*}
  y(k) &= T(p) \; u(k),
\end{align*}

with parameter $p$ that is time independent.  $y(k)$ denotes the
system output and $u(k)$ denotes the input.  The independent variable
$k$ can be denoted as time, index for a discrete sequences or
differential operaters (e.g. such as $s$ in Laplace domain or $\omega$
in frequency domain).

For example, for a simple mass-spring-dashpot system, the system
parameter $p$ can be selected as the mass $m$, spring constant $k$ and
damping coefficient $d$.  The input $u$ to the said system can be chosen
as the force applied to the mass and the output $y$ can be chosen as the
mass's displacement.

LTI system has the following properties.

\begin{description}
  \item[Linearity:]
    If $y_1 = T x_1$ and $y_2 = T x_2$, then 
    $$T \{\alpha x_1 + \beta x_2 \} = \alpha y_1 + \beta y_2 $$ 
  \item[Time Invariance:]
    If $y(k) = T x(k)$, then
    $$ y(k+\delta_k) = T x(k + \delta_k) $$
  \item[Associative:]
    $$ T_1 \cdot ( T_2 \cdot T_3 ) = (T_1 \cdot T_2) \cdot T_3 $$
  \item[Commutative:]
    $$ T_1 \cdot T_2 = T_2 \cdot T_1 $$
\end{description}

A LTI system can be represented with the following:

\begin{itemize}
  \item Transfer function of Laplace transform variable $s$, which is commonly
    used in control systems design.
  \item Transfer function of Fourier transform variable $\omega$, which is
    commonly used in communication theory and signal processing.
  \item Transfer function of z-transform variable $z^{-1}$, which is
    commonly used in digital signal processing (DSP).
  \item State-space equations, which is commonly used in modern control
    theory and mechanical systems.
\end{itemize}

Note that all transfer functions are LTI systems, but not all
state-space equations are LTI systems.
%%%%%
%%%%%
\end{document}
