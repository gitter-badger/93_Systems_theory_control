\documentclass[12pt]{article}
\usepackage{pmmeta}
\pmcanonicalname{StabilityOfTransferFunctionsInTheLaplaceDomain}
\pmcreated{2013-03-22 14:22:41}
\pmmodified{2013-03-22 14:22:41}
\pmowner{rrogers}{21140}
\pmmodifier{rrogers}{21140}
\pmtitle{stability of transfer functions in the Laplace domain}
\pmrecord{11}{35870}
\pmprivacy{1}
\pmauthor{rrogers}{21140}
\pmtype{Definition}
\pmcomment{trigger rebuild}
\pmclassification{msc}{93A10}
%\pmkeywords{BIBO stable}

\endmetadata

% this is the default PlanetMath preamble.  as your knowledge
% of TeX increases, you will probably want to edit this, but
% it should be fine as is for beginners.

% almost certainly you want these
\usepackage{amssymb}
\usepackage{amsmath}
\usepackage{amsfonts}

% used for TeXing text within eps files
%\usepackage{psfrag}
% need this for including graphics (\includegraphics)
%\usepackage{graphicx}
% for neatly defining theorems and propositions
%\usepackage{amsthm}
% making logically defined graphics
%%%\usepackage{xypic}

% there are many more packages, add them here as you need them

% define commands here
\begin{document}
The following describes SISO (single input single output) system descriptions.
More complex systems require a more sophisticated analysis.

For a general transfer function, $H(s)$, in Laplace domain, we have

\begin{align}
  \label{eq:tf}
   H(s) &= \frac{\prod_{i=0}^Z (s - z_i)}{\prod_{i=0}^P (s - p_i)}
\end{align}

The conclusions below can all be derived and understood by expansion of $H(s)$ in 
terms of partial fractions, and then doing a inverse Laplace 
transform fraction by fraction.

$z_i$ denotes the zeros and $p_i$ denotes the poles of the
linear time invariant system (LTI).  Stability of the system $H(s)$ is
characterized by the location of the poles in the complex s-plane.  
There are many definitions
of stability in the control system literature, the most common one
used (for transfer functions) is the bounded-input-bounded-output
stability (BIBO), which states that for a BIBO stable system, for any
bounded input, or finite amplitude input, the output of the system
will also be bounded. 

For example, a typical second order system such as the
mass-spring-dashpot system has the following transfer function,

\begin{align}
  \label{eq:2ndOrderSys}
  H(p) &= \frac{\omega^2}{p^2 + 2\zeta \omega p + \omega^2}
\end{align}

with a pair of complex poles at $p = -\zeta \omega \pm \omega
\sqrt{\zeta^2 -1}$.  In common control system literature, $\zeta$ is
usually denoted as the \emph{damping ratio} and $\omega$ is denoted as
the \emph{natural frequency} of the system.  In the case of the
mass-spring-dashpot system, we can tell that the oscillation of the
mass attached to a spring should be a function of the weight of the
mass and the stiffness of the spring, hence in the literature we have
$\omega = \sqrt{\frac{K}{M}}$, where $K$ is the spring constant and
$M$ is the weight of the mass.  From the same logic, the amount time
for the mass to stop oscillating should be a function of the spring
stiffness and the characteristics of the dashpot, so we have $\zeta =
\frac{D \omega}{2 K}$, where $D$ is the dashpot constant.  

To determine is the system $H(p)$ BIBO stable, the simplest solution
is to scrutinize the time domain solution of ~\ref{eq:2ndOrderSys} via
inverse Laplace transform, and such discussion can easily be found in
common control system related text books and online lecture notes.
However, such approach tells nothing about the \emph{physical nature}
of the dynamic system, so here I will try to establish relationship
between the location of the poles and the stability of the system.
Here I first state that the system in ~\ref{eq:2ndOrderSys},
specifically the mass-spring-dashpot system, is stable,
since it is physically impossible for the system to produce an output
increasing in amplitude forever as time goes to infinity.

First let's consider an ideal mass-spring system.  If we push the mass
once (impulse response), the mass will oscillate
forever with the same amplitude and frequency, since there are no
dashpot to dampen the motion.  In this case, $D=0 \rightarrow \zeta
=0$, which the pair of complex poles of the system will be located on
the imaginary axis of the complex s-plane, and the stronger the spring
($K$ is large) the further away the poles from the origin. So the
imaginary part of the poles $img(p_i)$ dominates the oscillatory nature of the
system. Now let's assume that we have a relatively weak spring compare
to the dashpot that we are going to add to the system, so the real
part of the poles $real(p_i)$ will be dominating.  If we push the mass
the same way as before, now we can expect the mass to oscillate for
sometime then come to a rest, depending on the strength of the
dashpot.  The stronger the dashpot, the further away the poles from
the origin along the real axis on the s-plane.  We have almost covered
the whole left-hand side of the s-plane, and so far we have the following
observations.

\begin{itemize}
  \item The imaginary part of complex poles correspond to oscillatory energy storage
    mechanisms. Note if the original differential equation has real coefficients, 
    the complex poles are always in complex conjugate pairs .  As a result
    these can always be represented by sine and cos terms in the time domain.
  \item The real parts of poles govern the decay rate of responses to 
    stimulations; with negative values corresponding to decaying terms, and positive
    parts to growth terms (which are not stable).
  \item The ratio between the real and imaginary parts of the poles govern the shape of the
   response.  In particular ratios less than one hide the visible part of the oscillatory terms in the
   time domain.
  \item Poles that have neither real or imaginary parts, i.e. at the origin, 
    correspond to successive integrations in the time domain.  One integration for 
    each pole.  Since these have an unbounded output for a step input, systems like these are 
    not BIBO; but are common.
\end{itemize}

For a system consisted of poles with imaginary parts only, $p_i = \pm
j \omega$, it is usually referred to as a \em{marginally stable
  system}. Notice that the real part of the poles for system
~\ref{eq:2ndOrderSys} are always negative, since both $\zeta$ and
$\omega$ are always positive. If a stable system has negative poles
(real or complex), unstable system must has positives poles, in the
right-hand side of the s-plane.  This can easily be visualize if we
place a amplifier (with $-\zeta$) instead of a dashpot in the system. 

%%%%%
%%%%%
\end{document}
