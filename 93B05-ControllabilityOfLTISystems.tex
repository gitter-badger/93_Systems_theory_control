\documentclass[12pt]{article}
\usepackage{pmmeta}
\pmcanonicalname{ControllabilityOfLTISystems}
\pmcreated{2013-03-22 14:32:50}
\pmmodified{2013-03-22 14:32:50}
\pmowner{GeraW}{6138}
\pmmodifier{GeraW}{6138}
\pmtitle{controllability of LTI systems}
\pmrecord{5}{36095}
\pmprivacy{1}
\pmauthor{GeraW}{6138}
\pmtype{Definition}
\pmcomment{trigger rebuild}
\pmclassification{msc}{93B05}
\pmdefines{controllability matrix}

\endmetadata

% this is the default PlanetMath preamble.  as your knowledge
% of TeX increases, you will probably want to edit this, but
% it should be fine as is for beginners.

% almost certainly you want these
\usepackage{amssymb}
\usepackage{amsmath}
\usepackage{amsfonts}

% used for TeXing text within eps files
%\usepackage{psfrag}
% need this for including graphics (\includegraphics)
%\usepackage{graphicx}
% for neatly defining theorems and propositions
%\usepackage{amsthm}
% making logically defined graphics
%%%\usepackage{xypic}

% there are many more packages, add them here as you need them

% define commands here
\begin{document}
Consider the linear time invariant (LTI) system given by:
\[
                   \dot{x} = Ax + Bu
\]
where $A$ is an $n \times n$ matrix, $B$ is an $n \times m$ matrix, $u$ is an $m \times 1$ vector - called the control or input vector, $x$ is an $n
\times 1$ vector - called the state vector, and $\dot{x}$ denotes the time derivative of $x$.


{\bf Definition Of Controllability Matrix For LTI Systems:} The controllability matrix of the above LTI system is defined by the pair $(A,B)$ as follows:
\[
             C(A,B) = \left[ B , AB,  A^{2}B,  A^{3}B,..., A^{n-1}B \right]
\]

{\bf Test for Controllability of LTI Systems:} The above LTI system $(A,B)$ is controllable if and only if the controllability matrix $C(A,B)$ has rank $n$;
i.e. has $n$ linearly independent columns.
%%%%%
%%%%%
\end{document}
