\documentclass[12pt]{article}
\usepackage{pmmeta}
\pmcanonicalname{ControlSystem}
\pmcreated{2013-03-22 14:22:59}
\pmmodified{2013-03-22 14:22:59}
\pmowner{ppirrip}{5555}
\pmmodifier{ppirrip}{5555}
\pmtitle{control system}
\pmrecord{9}{35876}
\pmprivacy{1}
\pmauthor{ppirrip}{5555}
\pmtype{Definition}
\pmcomment{trigger rebuild}
\pmclassification{msc}{93A10}
\pmrelated{SystemDefinitions}

\endmetadata

% this is the default PlanetMath preamble.  as your knowledge
% of TeX increases, you will probably want to edit this, but
% it should be fine as is for beginners.

% almost certainly you want these
\usepackage{amssymb}
\usepackage{amsmath}
\usepackage{amsfonts}

% used for TeXing text within eps files
%\usepackage{psfrag}
% need this for including graphics (\includegraphics)
%\usepackage{graphicx}
% for neatly defining theorems and propositions
%\usepackage{amsthm}
% making logically defined graphics
%%%\usepackage{xypic}

% there are many more packages, add them here as you need them

% define commands here
\def\u#1{\,\mathrm{#1}}
\begin{document}
The sole objective of control system is to generate feasible inputs
to the plant (e.g. dynamic systems) such that it will operate as it
intended to under a wide range of operating conditions.

Examples of control systems: Cruise control, auto pilot, rice cooker.

For a general finite dimensional dynamic system in its ODE form,

\begin{align}
  \label{eq:eq1}
  \dot{x} &= f(x(t),t) + g(x(t), u(t), t), \\
  y &= h(x(t),t), \nonumber
\end{align}

where $x \in \mathbb{R}^n$ is the state, $y \in \mathbb{R}^l$ is the
output and $u \in \mathbb{R}^m$ is the control input of the system.
In the control literature, equation \ref{eq:eq1} is general referred
as the plant, where the function $f: \mathbb{R}^n \times
\mathbb{R} \rightarrow \mathbb{R}^n$ governs the system dynamics, $g:
\mathbb{R}^m \times \mathbb{R}^n \times \mathbb{R} \rightarrow
\mathbb{R}^n$ determines how the input (control signals)
influence the state $x$ via {\em actuators} (e.g. gas turbine) and
 $h: \mathbb{R}^n \times \mathbb{R} \rightarrow \mathbb{R}^l$
 determines how the state generates the output signal.  If $m$ is equal to $n$, the plant is
{\em fully actuated}. If $m < n$ then the plant is {\em under actuated}
and otherwise the plant is {\em over actuated}. For a plant that is not
explicitly dependent in time $t$, such system is called a
{\em Autonomous} system.   The main
 differece between a control system and a general dynamic system is
 the additional signal $u(t)$.

For example, to control an airplane, the control system has to control
the {\em thrust}, {\em flaps}, {\em aileron} and {\em rudder}, which they
are the control signals of the system $u$.  Those control input
influence the system state $x$ such as {\em speed} (with thrust),
{\em attitude} and {\em orientation} (with flaps, aileron and rudder).
To physically alter the state of the airplane, actuators such as
gas turbines are needed, which are controlled by the control signals $u$.

The control signal $u$ can be generated in a {\em closed-loop} fashion
or {\em open-loop} fashion.  An open-loop control system generates $u$
with the user, or operator supplied reference state $x_{d}$ or output
$y_{d}$ only; meanwhile closed-loop control system uses both reference
and {\em feedback} signals that are usually measured from
{\em sensors}. In the airplane attitude control example, the desired
attitude is usually represented in roll-pitch-yaw angle
representation, and these signals are measurable by attaching
sensors to the flaps, aileron and rudder.  In engineering
practice, only closed-loop control systems should be used, since
open-loop systems are not {\em robust} against uncertainties, modeling
errors and measurement errors.

If a closed-loop control system is based on state feedback, such
contol system is called a {\em state-feedback} control system. By the
same token, a {\em output-feedback} control system is  based on output
feedback only.  Notice that output signals are available for feedback
by definition, however in reality not all the states are mesurable.  If
a state-feedback control system with all the states available for
feedback, it is called a {\em full-state feedback} system and otherwise
is call {\em partial-state feedback} system, which usually requires a {\em state
  observer} (e.g. Kalman filter) to estimate the unavailable states.

% -----------------------
To illustrate the simple concept of control systems, we will use a
simple example.  A truck driver is required to travel 1000 Km in 10
hours.  To relive the stress on the driver's heel, he has placed a stick
to the gas paddle so the car travels at $\u{100Km/h}$.  Under perfect
conditions, the driver will reach the destination in the allocated
time.  However, a certain section of the road is up-hill, so the truck
slowed down by a considerable amount and will not arrive it's
destination in time.  To remedy this problem, the driver 'implemented'
a simple solution using the speed-o-meter such that the gas paddle position
$p_{set}$ of the truck is now depends on the current speed
$v_{current}$ of the truck, $p_{set} = -K (v_{current} - \u{100Km/h})$,
where $K$ is just an adjustable parameter.  So if the truck is running
too slow (e.g. up-hill), $p_{set}$  will be positive (more gas to the
engine) hence speed will increase to maintain the desired speed, so
vice-versa for the down-hill case.  

%However, the *** is
%only measuring the engine's speed, not the real speed of the truck
%that is traveling. If there is a strong wind blowing against the
%truck, and it is strong enough to slow down the truck, the *** will not be
%able tell, and hence the truck will arrive late.

In this example, we have outlined all the major components of a
typical control system:

\begin{itemize}
  \item Actuator: engine,
  \item Sensor: speed-o-meter,
  \item Plant: truck,
  \item Control input: gas paddle,
  \item Control objective: 1000 Km in 10 hrs,
  \item Control law: $p_{set} = -K (v_{current} - 100Km/h)$,
  %\item Noise/uncertainity: strong wind.
\end{itemize}

The science aspect of control systems is the study of
design, synthesis and analysis of control systems using mathematical
concepts, and the engineering aspect of controls systems is to
implement, construct and adjust the control system according to
real-life situation and limitations.

%Special thanks to Prof. John Chen In Ryerson Polytechnic University.
%%%%%
%%%%%
\end{document}
