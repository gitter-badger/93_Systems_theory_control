\documentclass[12pt]{article}
\usepackage{pmmeta}
\pmcanonicalname{SystemModel}
\pmcreated{2013-03-22 18:33:34}
\pmmodified{2013-03-22 18:33:34}
\pmowner{camillio}{22337}
\pmmodifier{camillio}{22337}
\pmtitle{system model}
\pmrecord{6}{41282}
\pmprivacy{1}
\pmauthor{camillio}{22337}
\pmtype{Definition}
\pmcomment{trigger rebuild}
\pmclassification{msc}{93E03}
%\pmkeywords{Stochastic system}
%\pmkeywords{system identification}
%\pmkeywords{system model}

\endmetadata

% this is the default PlanetMath preamble.  as your knowledge
% of TeX increases, you will probably want to edit this, but
% it should be fine as is for beginners.

% almost certainly you want these
\usepackage{amssymb}
\usepackage{amsmath}
\usepackage{amsfonts}

% used for TeXing text within eps files
%\usepackage{psfrag}
% need this for including graphics (\includegraphics)
%\usepackage{graphicx}
% for neatly defining theorems and propositions
%\usepackage{amsthm}
% making logically defined graphics
%%%\usepackage{xypic}

% there are many more packages, add them here as you need them

% define commands here

\begin{document}
Let $t = 1, 2,\ldots$ denote discrete time instants. By a {\it system model} we mean a mathematical model defined by a conditional probability density function $f(y_{t}|u_{t}, d(t-1))$ where
\begin{description}
  \item [$y_{t}$] is the system output in time $t$,
  \item [$u_{t}$] is the system input and 
  \item [$d(t-1)$] denotes the sequence of data $d_{0}, \ldots, d_{t-1}$ where $d_{t} = (u_{t}, y_{t})$.
\end{description}
Such a system has time-invariant (constant) parameters. 
If the model parameters are unknown (uncertain, variable), we introduce the definition in the form $f(y_{t}|u_{t}, d(t-1), \theta)$. Here, $\theta$ is a (possibly multi-dimensional) parameter.

\begin{thebibliography}{99}
  \bibitem{vP81}
    Peterka, V., \emph{Bayesian Approach to System Identification}, in \emph{Trends and Progress in System Identification}, P. Ekhoff, Ed., pp. 239-304. Pergamon Press, Oxford, 1981
\end{thebibliography}
%%%%%
%%%%%
\end{document}
