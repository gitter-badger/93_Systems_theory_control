\documentclass[12pt]{article}
\usepackage{pmmeta}
\pmcanonicalname{TransferFunction}
\pmcreated{2013-03-22 14:02:41}
\pmmodified{2013-03-22 14:02:41}
\pmowner{lha}{3057}
\pmmodifier{lha}{3057}
\pmtitle{transfer function}
\pmrecord{13}{35394}
\pmprivacy{1}
\pmauthor{lha}{3057}
\pmtype{Definition}
\pmcomment{trigger rebuild}
\pmclassification{msc}{93A10}
\pmdefines{frequency domain}
\pmdefines{stable}
\pmdefines{unstable}

% this is the default PlanetMath preamble.  as your knowledge
% of TeX increases, you will probably want to edit this, but
% it should be fine as is for beginners.

% almost certainly you want these
\usepackage{amssymb}
\usepackage{amsmath}
\usepackage{amsfonts}

% used for TeXing text within eps files
%\usepackage{psfrag}
% need this for including graphics (\includegraphics)
%\usepackage{graphicx}
% for neatly defining theorems and propositions
%\usepackage{amsthm}
% making logically defined graphics
%%%\usepackage{xypic}

% there are many more packages, add them here as you need them

% define commands here
\begin{document}
The \emph{transfer function} of a linear dynamical system is the ratio of the Laplace transform of its output to the Laplace transform of its input.  In systems theory, the Laplace transform is called the ``frequency domain'' representation of the system.

Consider a canonical dynamical system
\begin{eqnarray*}
    \dot x(t) &=& A x(t) + B u(t) \\
    y (t) &=& C x(t) + D u(t)
\end{eqnarray*}
with input $u: R \mapsto R^n$, output $y: R \mapsto R^m$ and state $x:R \mapsto R^p$, and $(A,B,C,D)$ are constant matrices of conformable sizes.

The frequency domain representation is
$$
    y(s) = (D + C(sI - A)^{-1}B)u(s),
$$
and thus the transfer function matrix is $D + C(sI - A)^{-1}B$.

In the case of single-input-single-output systems ($m=n=1$), the transfer function is commonly expressed as a rational function of $s$:
$$
    H(s) = \frac{\prod_{i=0}^Z (s - z_i)}{\prod_{i=0}^P (s - p_i)}.
$$
The values $z_i$ are called the zeros of $H(s)$, and the values $p_i$ are called the poles.  If any of the poles has positive real part, then the transfer function is termed \emph{unstable}; if all of the poles have strictly negative real part, it is \emph{stable}.
%%%%%
%%%%%
\end{document}
